%
% Chapter 9
%

\chapter{Results}
\label{results}

After applying the selection criteria, the expected limits are calculated from a final discrimination variable using the profile likelihood method with asymptotic approximation. Two different fits are considered, with differences in the discrimination variable employed and changes to the selection. The primary results are obtained from a fit to the BDT discriminator after the loose selection. The cross-check results are obtained from a fit to the \mcol distribution after applying cut-based selection criteria. The fits are performed per channel and category and then combined to set 95\% CL upper limits on the branching fraction of LFV Higgs decay in the \Hmt and \Het channels, \BHmt, and \BHet, respectively. The BDT discriminator distributions of signal and background for each category for \Hmt and \Het can be seen in Figures ~\ref{fig:bdt_muhad}, ~\ref{fig:bdt_mue}, ~\ref{fig:bdt_ehad}, and ~\ref{fig:bdt_emu} respectively. The $\mcol$ discriminator distributions of signal and background for each category for \Hmt and \Het can be seen in Figures ~\ref{fig:mcol_muhad}, ~\ref{fig:mcol_mue}, ~\ref{fig:mcol_ehad}, and ~\ref{fig:mcol_emu} respectively.

\begin{figure}[htbp!]
  \centering
  \includegraphics[width=0.4\textwidth]{plots/chapter9/BDT/mutau/0jet.png}
  \includegraphics[width=0.4\textwidth]{plots/chapter9/BDT/mutau/1jet.png} \\
  \includegraphics[width=0.4\textwidth]{plots/chapter9/BDT/mutau/2jet_gg.png}
  \includegraphics[width=0.4\textwidth]{plots/chapter9/BDT/mutau/2jet_vbf.png} \\
  \caption{BDT discriminator distributions for the observed and estimated background in the \Hmuhad process. The background is normalized to the best fit values from the signal plus background fit. Signal corresponds to \BHmt = 5\%. \Hmuhad channel categories are 0 jets (top left), 1 jet (top right), 2 jets gg (bottom left), and 2 jets VBF (bottom right). The bottom panel in each plot shows the fractional difference between the observed and estimated background. The uncertainty band shows the post fit statistical and systematic uncertainties added in quadrature.}
  \label{fig:bdt_muhad}
\end{figure}

\begin{figure}[htbp!]
  \centering
  \includegraphics[width=0.4\textwidth]{plots/chapter9/BDT/mue/0jet.png}
  \includegraphics[width=0.4\textwidth]{plots/chapter9/BDT/mue/1jet.png} \\
  \includegraphics[width=0.4\textwidth]{plots/chapter9/BDT/mue/2jet_gg.png}
  \includegraphics[width=0.4\textwidth]{plots/chapter9/BDT/mue/2jet_vbf.png} \\
  \caption{BDT discriminator distributions for the observed and estimated background in the \Hmue process. The background is normalized to the best fit values from the signal plus background fit. Signal corresponds to \BHmt = 5\%. \Hmue channel categories are 0 jets (top left), 1 jet (top right), 2 jets gg (bottom left), and 2 jets VBF (bottom right). The bottom panel in each plot shows the fractional difference between the observed and estimated background. The uncertainty band shows the post fit statistical and systematic uncertainties added in quadrature.}
  \label{fig:bdt_mue}
\end{figure}

\begin{figure}[htbp!]
  \centering
  \includegraphics[width=0.4\textwidth]{plots/chapter9/CB/mutau/0jet.png}
  \includegraphics[width=0.4\textwidth]{plots/chapter9/CB/mutau/1jet.png} \\
  \includegraphics[width=0.4\textwidth]{plots/chapter9/CB/mutau/2jet_gg.png}
  \includegraphics[width=0.4\textwidth]{plots/chapter9/CB/mutau/2jet_vbf.png} \\
  \caption{\mcol distributions for the observed and estimated background in the \Hmuhad process. The background is normalized to the best fit values from the signal plus background fit. Signal corresponds to \BHmt = 10\%. \Hmuhad channel categories are 0 jets (top left), 1 jet (top right), 2 jets gg (bottom left), and 2 jets VBF (bottom right). The bottom panel in each plot shows the fractional difference between the observed and estimated background. The uncertainty band shows the post fit statistical and systematic uncertainties added in quadrature.}
  \label{fig:mcol_muhad}
\end{figure}

\begin{figure}[htbp!]
  \centering
  \includegraphics[width=0.4\textwidth]{plots/chapter9/CB/mue/0jet.png}
  \includegraphics[width=0.4\textwidth]{plots/chapter9/CB/mue/1jet.png} \\
  \includegraphics[width=0.4\textwidth]{plots/chapter9/CB/mue/2jet_gg.png}
  \includegraphics[width=0.4\textwidth]{plots/chapter9/CB/mue/2jet_vbf.png} \\
  \caption{\mcol distributions for the observed and estimated background in the \Hmue process. The background is normalized to the best fit values from the signal plus background fit. Signal corresponds to \BHmt = 10\%. \Hmue channel categories are 0 jets (top left), 1 jet (top right), 2 jets gg (bottom left), and 2 jets VBF (bottom right). The bottom panel in each plot shows the fractional difference between the observed and estimated background. The uncertainty band shows the post fit statistical and systematic uncertainties added in quadrature.}
  \label{fig:mcol_mue}
\end{figure}

\begin{figure}[htbp!]
  \centering
  \includegraphics[width=0.4\textwidth]{plots/chapter9/BDT/etau/0jet.png}
  \includegraphics[width=0.4\textwidth]{plots/chapter9/BDT/etau/1jet.png} \\
  \includegraphics[width=0.4\textwidth]{plots/chapter9/BDT/etau/2jet_gg.png}
  \includegraphics[width=0.4\textwidth]{plots/chapter9/BDT/etau/2jet_vbf.png} \\
  \caption{BDT discriminator distributions for the observed and estimated background in the \Hehad process. The background is normalized to the best fit values from the signal plus background fit. Signal corresponds to \BHet = 5\%. \Hehad channel categories are 0 jets (top left), 1 jet (top right), 2 jets gg (bottom left), and 2 jets VBF (bottom right). The bottom panel in each plot shows the fractional difference between the observed and estimated background. The uncertainty band shows the post fit statistical and systematic uncertainties added in quadrature.}
  \label{fig:bdt_ehad}
\end{figure}

\begin{figure}[htbp!]
  \centering
  \includegraphics[width=0.4\textwidth]{plots/chapter9/BDT/emu/0jet.png}
  \includegraphics[width=0.4\textwidth]{plots/chapter9/BDT/emu/1jet.png} \\
  \includegraphics[width=0.4\textwidth]{plots/chapter9/BDT/emu/2jet_gg.png}
  \includegraphics[width=0.4\textwidth]{plots/chapter9/BDT/emu/2jet_vbf.png} \\
  \caption{BDT discriminator distributions for the observed and estimated background in the \Hemu process. The background is normalized to the best fit values from the signal plus background fit. Signal corresponds to \BHet = 5\%. \Hemu channel categories are 0 jets (top left), 1 jet (top right), 2 jets gg (bottom left), and 2 jets VBF (bottom right). The bottom panel in each plot shows the fractional difference between the observed and estimated background. The uncertainty band shows the post fit statistical and systematic uncertainties added in quadrature.}
  \label{fig:bdt_emu}
\end{figure}

\begin{figure}[htbp!]
  \centering
  \includegraphics[width=0.4\textwidth]{plots/chapter9/CB/etau/0jet.png}
  \includegraphics[width=0.4\textwidth]{plots/chapter9/CB/etau/1jet.png} \\
  \includegraphics[width=0.4\textwidth]{plots/chapter9/CB/etau/2jet_gg.png}
  \includegraphics[width=0.4\textwidth]{plots/chapter9/CB/etau/2jet_vbf.png} \\
  \caption{\mcol distributions for the observed and estimated background in the \Hehad process. The background is normalized to the best fit values from the signal plus background fit. Signal corresponds to \BHet = 10\%. \Hehad channel categories are 0 jets (top left), 1 jet (top right), 2 jets gg (bottom left), and 2 jets VBF (bottom right). The bottom panel in each plot shows the fractional difference between the observed and estimated background. The uncertainty band shows the post fit statistical and systematic uncertainties added in quadrature.}
  \label{fig:mcol_ehad}
\end{figure}

\begin{figure}[htbp!]
  \centering
  \includegraphics[width=0.4\textwidth]{plots/chapter9/CB/emu/0jet.png}
  \includegraphics[width=0.4\textwidth]{plots/chapter9/CB/emu/1jet.png} \\
  \includegraphics[width=0.4\textwidth]{plots/chapter9/CB/emu/2jet_gg.png}
  \includegraphics[width=0.4\textwidth]{plots/chapter9/CB/emu/2jet_vbf.png} \\
  \caption{\mcol distributions for the observed and estimated background in the \Hemu process. The background is normalized to the best fit values from the signal plus background fit. Signal corresponds to \BHet = 10\%. \Hemu channel categories are 0 jets (top left), 1 jet (top right), 2 jets gg (bottom left), and 2 jets VBF (bottom right). The bottom panel in each plot shows the fractional difference between the observed and estimated background. The uncertainty band shows the post fit statistical and systematic uncertainties added in quadrature.}
  \label{fig:mcol_emu}
\end{figure}

The BDT analysis yields a sensitivity on the Higgs branching fraction of 0.15\% (0.16\%) for \Hmt and 0.29\% (0.19\%) for \Het. The observed and median expected 95\% CL upper limits and the best fit branching fractions, for \BHmt and \BHet, assuming $\mh = 125 \GeV$, are reported in Tables ~\ref{tab:limit_bdt_mutau} and ~\ref{tab:limit_bdt_etau}. The limits are also summarized graphically in Figure ~\ref{fig:bdt_limits} and in Table ~\ref{tab:limits_summary}.

A cross-check analysis in which a maximum-likelihood fit is performed on the \mcol distribution after applying additional selections ~\cite{Sirunyan:2017xzt} yields a sensitivity on the Higgs branching fraction of 0.38\% (0.26\%) for \Hmt and 0.32\% (0.28\%) for \Het. The observed and median expected 95\% CL upper limits and the best fit branching fractions, for \BHmt and \BHet, assuming $\mh = 125 \GeV$, are reported in Tables ~\ref{tab:limit_cb_mutau} and ~\ref{tab:limit_cb_etau}. The limits are also summarized graphically in Figure ~\ref{fig:cb_limits}. The BDT fit analysis is more sensitive than the \mcol fit analysis, and systematic uncertainties dominate results for both cases.

The upper limits on \BHmt and \BHet are subsequently used to put constraints on LFV Yukawa couplings ~\cite{Harnik:2012pb}. The LFV decays \Pe{}\Pgt\, and \Pgm{}\Pgt\, arise at tree level from the assumed flavor violating Yukawa interactions, $Y_{\ell^\alpha\ell^{\beta}}$, where $\ell^\alpha, \ell^\beta$ are the leptons (\Pe, \Pgm, \Pgt) of different flavors ($\alpha\ne\beta$). The decay width $\Gamma(\PH \to \ell^{\alpha} \ell^{\beta})$ in terms of the Yukawa couplings is given by:

\[\Gamma(\PH \to \ell^{\alpha} \ell^{\beta}) = \frac{m_{\PH}}{8\pi}(\abs{Y_{\ell^{\alpha}\ell^{\beta}}}^2 + \abs{Y_{\ell^{\beta}\ell^{\alpha}}}^2),\]
and the branching fractions is given by:
\[\mathcal{B}(\PH \to \ell^{\alpha} \ell^{\beta}) = \frac{\Gamma(\PH \to \ell^{\alpha} \ell^{\beta})}{\Gamma(H \to \ell^{\alpha} \ell^{\beta}) + \Gamma_{\text{SM}}}.\]

The SM \PH\, decay width is assumed to be $\Gamma_{\text{SM}} = 4.1\MeV$ ~\cite{Denner:2011mq} for $m_\PH = 125\GeV$. The 95\% CL upper limit on the Yukawa couplings derived from the expression for the branching fraction above is shown in Table ~\ref{tab:limits_summary}. The limits on the Yukawa couplings derived from the BDT fit analysis results are shown in Figure ~\ref{fig:bdt_yukawa_limits}.

%%-----------------------------------
%% Limits:
%%-----------------------------------
\begin{figure}[htbp!]
  \centering
  \includegraphics[width=0.45\textwidth]{plots/chapter9/limits/BDTMu.pdf}
  \includegraphics[width=0.45\textwidth]{plots/chapter9/limits/BDTE.pdf} \\
  \caption{Observed (expected) 95\% CL upper limits on the \BHmt (left) and \BHet (right) for each individual category and combined from the BDT fit analysis.}
  \label{fig:bdt_limits}
\end{figure}

% \begin{figure}[htbp!]
%   \centering
%   \includegraphics[width=0.45\textwidth]{plots/chapter9/limits/CBMu.pdf}
%   \includegraphics[width=0.45\textwidth]{plots/chapter9/limits/CBE.pdf} \\
%   \caption{Observed (expected) 95\% CL upper limits on the \BHmt (left) and \BHet (right) for each individual category and combined from the \mcol fit analysis.}
%   \label{fig:cb_limits}
% \end{figure}

%%-----------------------------------
%% Yukawa Limits:
%%-----------------------------------
\begin{figure}[htbp!]
  \centering
  \includegraphics[width=0.45\textwidth]{plots/chapter9/limits/Ymt.pdf}
  \includegraphics[width=0.45\textwidth]{plots/chapter9/limits/Yet.pdf} \\
  \caption{Constraints on the LFV Yukawa couplings, $\Ymutau-\Ytaumu$ (left), and $\Yetau-\Ytaue$ (right). The expected (red line) and observed (black solid line) limits are derived from the results shown in Figure ~\ref{fig:bdt_limits}. The flavor-diagonal Yukawa couplings are approximated by their SM values. The green hashed region is derived by the CMS direct search presented in this paper. The green (yellow) band indicates the range that is expected to contain 68\% (95\%) of all observed limit variations from the expected limit. The shaded regions are derived constraints from null searches for $\Pgt\to3\Pgm$ or $\Pgt\to3\Pe$ (dark green) ~\cite{Hayasaka:2010np} and $\Pgt\to\Pgm\Pgg$ or $\Pgt\to\Pe\Pgg$ (lighter green) ~\cite{Harnik:2012pb}. The blue diagonal line is the theoretical naturalness limit $|Y_{ij}Y_{ji}|\leq{m_i}m_j/v^2$.}
  \label{fig:bdt_yukawa_limits}
\end{figure}

\begin{table}[!hbpt]
\centering
\caption{Observed and expected upper limits at 95\% CL and best fit branching fractions for each individual category, and combined, in the \Hmt process from BDT fit analysis.}
\begin{tabular}{cccccc}
\hline
\multicolumn{6}{c}{Expected limits (\%)}                  \\
\hline
       & 0-jet   & 1-jet   & 2-jets  & VBF     & Combined \\
\cline{2-6}
\mue   & $<0.34$ & $<0.57$ & $<1.13$ & $<0.83$ & $<0.27$  \\
\muhad & $<0.33$ & $<0.43$ & $<0.49$ & $<0.30$ & $<0.18$  \\
\cline{2-6}
\mutau & \multicolumn{5}{c}{$<0.15$}                      \\
\hline
\multicolumn{6}{c}{Observed limits (\%)}                  \\
\hline
       & 0-jet   & 1-jet   & 2-jets  & VBF     & Combined \\
\cline{2-6}
\mue   & $<0.31$ & $<0.36$ & $<0.77$ & $<0.58$ & $<0.19$  \\
\muhad & $<0.37$ & $<0.40$ & $<0.50$ & $<0.39$ & $<0.24$  \\
\cline{2-6}
\mutau & \multicolumn{5}{c}{$<0.15$}                      \\
\hline
\multicolumn{6}{c}{Best fit branching fractions (\%)}                                       \\
\hline
       & 0-jet          & 1-jet          & 2-jets         & VBF            & Combined       \\
\cline{2-6}
\mue   & $-0.03\pm0.17$ & $-0.40\pm0.28$ & $-0.66\pm0.56$ & $-0.41\pm0.39$ & $-0.14\pm0.13$ \\
\muhad & $0.05\pm0.17$  & $-0.05\pm0.22$ & $0.02\pm0.25$  & $0.10\pm0.16$  & $0.07\pm0.09$  \\
\cline{2-6}
\mutau & \multicolumn{5}{c}{$0.00\pm0.07$}                                                  \\
\hline
\end{tabular}
\label{tab:limit_bdt_mutau}
\end{table}

\begin{table}[!hbpt]
\centering
\caption{Observed and expected upper limits at 95\% CL and best fit branching fractions for each individual category, and combined, in the \Het process from BDT fit analysis.}
\begin{tabular}{cccccc}
\hline
\multicolumn{6}{c}{Expected limits (\%)}                 \\
\hline
      & 0-jet   & 1-jet   & 2-jets  & VBF     & Combined \\
\cline{2-6}
\emu  & $<0.34$ & $<0.53$ & $<1.08$ & $<0.86$ & $<0.26$  \\
\ehad & $<0.39$ & $<0.44$ & $<0.55$ & $<0.35$ & $<0.20$  \\
\cline{2-6}
\etau & \multicolumn{5}{c}{$<0.16$}                      \\
\hline
\multicolumn{6}{c}{Observed limits (\%)}                 \\
\hline
      & 0-jet   & 1-jet   & 2-jets  & VBF     & Combined \\
\cline{2-6}
\emu  & $<0.42$ & $<0.56$ & $<1.35$ & $<0.42$ & $<0.22$  \\
\ehad & $<0.44$ & $<0.68$ & $<0.78$ & $<0.57$ & $<0.37$  \\
\cline{2-6}
\etau & \multicolumn{5}{c}{$<0.22$}                      \\
\hline
\multicolumn{6}{c}{Best fit branching fractions (\%)}                                    \\
\hline
      & 0-jet         & 1-jet          & 2-jets        & VBF            & Combined       \\
\cline{2-6}
\emu  & $0.11\pm0.17$ & $0.04\pm0.27$  & $0.35\pm0.55$ & $-1.04\pm0.44$ & $-0.07\pm0.13$ \\
\ehad & $0.07\pm0.20$ & $0.29\pm0.23$  & $0.27\pm0.29$ & $0.27\pm0.17$  & $0.20\pm0.10$  \\
\cline{2-6}
\etau & \multicolumn{5}{c}{$0.08\pm0.08$}                                                \\
\hline
\end{tabular}
\label{tab:limit_bdt_etau}
\end{table}

\begin{table}[!hbpt]
\centering
\caption{Observed and expected upper limits at 95\% CL and best fit branching fractions for each individual jet category, and combined, in the \Hmt process from \mcol fit analysis.}
\begin{tabular}{cccccc}
\hline
\multicolumn{6}{c}{Expected limits (\%)}                           \\
\hline
       & 0-jet     & 1-jet     & 2-jets    & VBF       & Combined  \\
\cline{2-6}
\mue   & $<$ 0.55  & $<$ 0.81  & $<$ 1.70  & $<$ 1.18  & $<$ 0.42  \\
\muhad & $<$ 0.60  & $<$ 0.54  & $<$ 0.94  & $<$ 0.65  & $<$ 0.34  \\
\cline{2-6}
\mutau & \multicolumn{5}{c}{$<$ 0.26}                              \\
\hline
\multicolumn{6}{c}{Observed limits (\%)}                           \\
\hline
       & 0-jet     & 1-jet     & 2-jets    & VBF       & Combined  \\
\cline{2-6}
\mue   & $<$ 0.93  & $<$ 0.88  & $<$ 2.12  & $<$ 0.92  & $<$ 0.60  \\
\muhad & $<$ 0.39  & $<$ 1.07  & $<$ 0.92  & $<$ 0.85  & $<$ 0.46  \\
\cline{2-6}
\mutau & \multicolumn{5}{c}{$<$ 0.38}                              \\
\hline
\multicolumn{6}{c}{Best fit branching fractions (\%)}              \\
\hline
      & 0-jet     & 1-jet     & 2-jets    & VBF       & Combined   \\
\cline{2-6}
\mue   & 0.45 $\pm$ 0.29  & 0.08 $\pm$ 0.42  & 0.56 $\pm$ 0.85  & -0.35 $\pm$ 0.55 & 0.23 $\pm$ 0.21 \\
\muhad & -0.42 $\pm$ 0.32 & 0.62 $\pm$ 0.27  & -0.02 $\pm$ 0.48 & 0.29 $\pm$ 0.33  & 0.16 $\pm$ 0.17 \\
\cline{2-6}
\mutau & \multicolumn{5}{c}{ 0.15 $\pm$ 0.13 }                                                       \\
\hline
\end{tabular}
\label{tab:limit_cb_mutau}
\end{table}

\begin{table}[!hbpt]
\centering
\caption{Observed and expected upper limits at 95\% CL and best fit branching fractions for each individual category, and combined, in the \Het process from \mcol fit analysis.}
\begin{tabular}{cccccc}
\hline
\multicolumn{6}{c}{Expected limits (\%)}                          \\
\hline
      & 0-jet     & 1-jet     & 2-jets    & VBF       & Combined  \\
\cline{2-6}
\emu  & $<$ 0.56  & $<$ 0.79  & $<$ 1.67  & $<$ 1.35  & $<$ 0.44  \\
\ehad & $<$ 0.67  & $<$ 0.65  & $<$ 1.05  & $<$ 0.71  & $<$ 0.38  \\
\cline{2-6}
\etau & \multicolumn{5}{c}{$<$ 0.28}                              \\
\hline
\multicolumn{6}{c}{Observed limits (\%)}                          \\
\hline
      & 0-jet     & 1-jet     & 2-jets    & VBF       & Combined  \\
\cline{2-6}
\emu  & $<$ 1.44  & $<$ 0.82  & $<$ 1.29  & $<$ 0.67  & $<$ 0.65  \\
\ehad & $<$ 0.47  & $<$ 0.67  & $<$ 1.15  & $<$ 0.83  & $<$ 0.34  \\
\cline{2-6}
\etau & \multicolumn{5}{c}{$<$ 0.32}                              \\
\hline
\multicolumn{6}{c}{Best fit branching fractions (\%)}             \\
\hline
      & 0-jet     & 1-jet     & 2-jets    & VBF       & Combined  \\
\cline{2-6}
\emu  & 0.96 $\pm$ 0.29  & 0.04 $\pm$ 0.40  & -0.66 $\pm$ 0.84 & -1.64 $\pm$ 0.68 & 0.26 $\pm$ 0.23  \\
\ehad & -0.45 $\pm$ 0.39 & 0.04 $\pm$ 0.34  & 0.14 $\pm$ 0.53  & 0.17 $\pm$ 0.37  & -0.08 $\pm$ 0.20 \\
\cline{2-6}
\etau & \multicolumn{5}{c}{0.05 $\pm$ 0.14}                                                          \\
\hline
\end{tabular}
\label{tab:limit_cb_etau}
\end{table}

\begin{table}
\centering
\caption{Summary of observed and expected upper limits at 95\% CL, best fit branching fractions and corresponding constaints on Yukawa couplings for \Hmt and \Het processes.}
\begin{tabular}{lcc}
\hline
     & Observed (Expected) upper limits (\%) & Best fit branching fractions (\%)  \\
\cline{2-3}
\Hmt & $\leq$ 0.15 (0.15)                    & 0.00 $\pm$ 0.07                   \\
\Het & $\leq$ 0.22 (0.16)                    & 0.08 $\pm$ 0.08                    \\
\hline
     & \multicolumn{2}{c}{Yukawa coupling constraints}                            \\
\cline{2-3}
\Hmt & \multicolumn{2}{c}{$\leq$ 1.11 (1.10)$\times 10^{-3}$}                     \\
\Het & \multicolumn{2}{c}{$\leq$ 1.35 (1.14)$\times 10^{-3}$}                     \\
\hline
\end{tabular}
\label{tab:limits_summary}
\end{table}

