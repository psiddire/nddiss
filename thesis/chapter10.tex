%
% Chapter 10
%

\chapter{Conclusion}
\label{conclusion}

In this dissertation, the search for SM Higgs boson decaying into LFV decays into a muon and tau or an electron and tau has been presented. The search is performed with the full Run 2 data collected at the CMS experiment in 2016, 2017, and 2018 at a center-of-mass energy of $13 \TeV$ corresponding to an integrated luminosity of $137 \fb$. The search found no evidence of the LFV decays of the Higgs boson, and corresponding exclusion upper limits have been placed on the branching fraction of Higgs boson into the \mutau and \etau final states.

The analysis has been completed and is currently going through the CMS collaboration's approval process and is expected to be published in the Journal for High Energy Physics. The observed (median expected) upper limits on \BHmt is 0.15 (0.16)\,\% at 95\% CL and on \BHet is 0.29 (0.19)\,\% at 95\% CL. The limits on the branching fraction have been correspondingly translated into limits on the off-diagonal Yukawa coupling and are set to be $\sqrt{\Ymutau^{2}+\Ytaumu^{2}} < 1.12 \times 10^{-3}$ and $\sqrt{\Yetau^{2}+\Ytaue^{2}} < 1.55 \times 10^{-3}$. These are the most stringent limits set in the LFV Higgs decays to date and constitute a significant improvement from the previous results.

The results are a factor of two improvement over the results published with just the 2016 dataset. Several improvements are performed compared to the previous search, and the various details have been explained in this dissertation to a good extent. In summary, using the DeepTau identification for the hadronically decaying tau leptons has significantly improved the sensitivity of the search in the hadronic channels. In the leptonic channels, the QCD background estimation technique has been updated. \Ztt background is estimated using the embedding technique for all channels and helps provide a good event description and better control over the systematic uncertainties. The misidentified lepton background in the hadronic channels has been modified. New shape systematics that weren't considered in the 2016 analysis are introduced to account for any residual discrepancies between the data and estimated backgrounds. All these improvements have compounded to set the most stringent limits on these branching fractions to date.

The LHC is expected to start taking data for Run 3 starting from the end of 2021, and currently, the expectation is that the center-of-mass energy will remain the same as for Run 2, i.e., $13 \TeV$. We anticipate collecting approximately $150 \fb$ in Run 3, and this will double the available dataset for Physics analysis. For the next iteration of this search, a new signal mass variable is proposed and is presented in the appendix (``Classic'' SVFit mass). From an initial study, this variable has been shown to improve the signal's mass resolution by approximately 20\%, which is a significant improvement. Any future iteration of this search is highly recommended to use this mass variable for the signal mass hypothesis to improve the analysis's sensitivity.
