%
% Chapter 1
%

\chapter{Introduction}
\label{introduction}

Our universe's nature is explored in particle physics by understanding elementary particles' properties and the fundamental forces. The fundamental forces are mediated by the exchange of particles, called gauge bosons. The \PW and \PZ bosons mediate the weak interaction. These bosons are heavy, and they acquire mass through the Higgs mechanism. The Higgs mechanism is incorporated in the standard model (SM) by introducing a new scalar field associated with a new particle, the Higgs boson. The SM has been well tested over a wide range of energies, and despite its success, it is considered an effective theory as it fails to include gravity. It also cannot explain dark matter and dark energy, which are inferred from astrophysical/cosmological observations and our universe's accelerated expansion~\cite{Ade:2015xua}. Also, the CP-violating effects of the SM cannot explain the large difference between matter and anti-matter in the universe.

The flavor is a term used to describe the different kinds of leptons (electron, muon, or tau). A lepton flavor violation (LFV) is a transition that does not conserve the lepton flavor. The neutrino oscillations have been experimentally observed at several experiments (Super-Kamiokande collaboration~\cite{Fukuda:1998mi} and Sudbury Neutrino Observatory collaboration~\cite{Ahmad:2002jz}), suggesting that LFV occurs in the neutral lepton sector (electron neutrino, muon neutrino, or tau neutrino) and that the neutrinos mix and have a mass. However, no charged LFV has been observed to date. Yukawa interaction is an interaction between a scalar field (or pseudoscalar field) and a Dirac field. The Yukawa interactions within the SM cannot be constrained by gauge invariance. Yukawa interactions between the scalar field and the fermion fields give rise to the mass of the fermions. In the SM, the mass matrix and the Yukawa interaction matrix are diagonalized simultaneously by choosing a particular eigenstate basis as they have the same coefficient in the Lagrangian. This corresponds to the SM Higgs boson not having LFV decays.

Flavor physics is an active area of research as we try to understand the reason for observing six quarks and six fermions that can be arranged into three generations. The Cabibbo-Kobayashi-Maskawa (CKM) matrix is the observed CP-violation source, a $3 \times 3$ unitary matrix that can be parameterized by three mixing angles and one complex CP-violating phase~\cite{Tanabashi:2018oca}. The CKM matrix describes the quarks' mixing because the mass eigenstates are not equal to the weak eigenstates.

A baryon is a type of composite subatomic particle that contains an odd number of valence quarks. A baryon number is a quantum number equal to the number of baryons in a system of subatomic particles minus the number of antibaryons. A lepton number is a quantum number representing the difference between the number of leptons and the number of antileptons in an elementary particle reaction.  In the SM, due to the absence of right-handed neutrinos, there is no Dirac mass term for neutrinos. In the SM, due to the exact conservation of baryon and lepton number, there is no Majorana mass term. However, neutrino oscillations have been observed, suggesting that neutrinos have a mass. The observation of right-handed neutrinos or violation of lepton number can explain this mass term. The mixing of the neutrinos is described by the Pontecorvo-Maki-Nakagawa-Sakata (PMNS) matrix~\cite{Tanabashi:2018oca}. An additional source of the CP-violation is the PMNS matrix described by three mixing angles and a phase. Besides, the SM cannot explain the substantial difference between the very small neutrino masses ($< 2~\eV$ for the electron neutrino) and the masses of the charged leptons and quarks ($\sim 173~\GeV$ for the top quark)~\cite{Tanabashi:2018oca}. One question that needs to be addressed is any relation between quark mixing and neutrino mixing.

To address the open questions, we need new physics models. Supersymmetry is a possible extension of the SM that could resolve the mass hierarchy problems within gauge theory by guaranteeing that quadratic divergences of all orders will cancel out in perturbation theory. Grand unification theories are a class of new physics models that try to unify all known interactions but gravity in a single gauge group. At the same time, extra dimensions could link flavor to the geometry of these extra dimensions~\cite{Raidal:2008jk}. Another class of new physics models is simple extensions of the SM with more than one Higgs doublets, and they could add additional sources for CP-violation. These multi-Higgs doublet models can have tree level Higgs boson mediated flavor changing neutral currents leading to LFV Yukawa couplings~\cite{Raidal:2008jk}.

The Large Hadron Collider (LHC)~\cite{Evans:2008zzb} was built to discover the Higgs boson and other exotic particles predicted by models like Supersymmetry or potential dark matter particles like weakly interacting massive particles. It was designed to run at a center-of-mass energy of 14~\TeV and is the most powerful particle collider built and operational to date. ATLAS and Compact Muon Solenoid (CMS) are two general-purpose detectors designed to detect the result of the particle collisions. In 2012, both experiments discovered the Higgs boson with a mass of 125~\GeV~\cite{Aad:2012tfa, Chatrchyan:2012ufa}. Many precision measurements need to be performed to confirm if it is the SM Higgs boson or a Higgs boson of an SM's possible extension. Until now, no significant deviations from the SM Higgs boson have been observed.

In this thesis, a search for the Higgs boson's LFV decays to a muon and a tau (\mutau) or an electron and a tau (\etau) is presented. In each channel, the tau can further decay either hadronically or leptonically. If the tau lepton decays leptonically, we only consider the final states with different lepton flavors to avoid the large Drell-Yan background. The Drell-Yan process occurs when a quark of one hadron and an antiquark of another hadron annihilate, creating a virtual photon or \PZ boson, decaying into a pair of oppositely charged leptons. Thus, the \mutau channel is further divided into \muhad and \mue final states, while the \etau channel is further divided into the \ehad and \emu final states. This search is performed with proton-proton (\pp) collision data collected at a center-of-mass energy of 13~\TeV in 2016, 2017, and 2018 corresponding to an integrated luminosity of 35.9~\fb, 41.5~\fb, and 59.3~\fb, respectively. Thus, the total integrated luminosity analyzed in this search is 137~\fb, which corresponds to a four times larger dataset than the one used in the previous searches~\cite{Sirunyan:2017xzt, Aad:2019ugc}.

Apart from the much larger dataset, significant improvements have been made regarding the background estimation techniques and a detailed study of the systematic uncertainties involved in the analysis. A significant portion of the background is estimated using data-driven techniques with limited dependence on the Monte Carlo (MC) simulations. This gives rise to a better description of the event kinematics and an improvement regarding the corresponding systematics involved. The categorization of the events has been kept the same as the previous search. In contrast, improvements have been made to the classification done with multivariate techniques for discriminating the signal from the background to improve the search's sensitivity. All these changes collectively gave rise to an improvement in the sensitivity of the search and helped to set the most stringent limits on these LFV Higgs decays to date.

This thesis is structured as follows. An overview of the SM of particle physics, along with a short review on LFV decays of the Higgs boson, is given in Chapter~\ref{theory}. The experimental setup of LHC and the CMS experiment is discussed in Chapter~\ref{experiment}. In Chapter~\ref{event_sim}, we will dive into the MC event generation, followed by the event reconstruction description in Chapter~\ref{event_reco}. The event selection is described in Chapter~\ref{event_sel}, followed by a detailed explanation of the background estimation in Chapter~\ref{bkg_est}. We will discuss the systematic uncertainties in Chapter~\ref{syst_unc} and the corresponding statistical analysis to obtain the results discussed in Chapter~\ref{results}, and finally concluding in Chapter~\ref{conclusion}. Some studies for the future LFV analysis have been performed, and they are detailed in the appendix~\ref{SVFit} and appendix~\ref{fakefactor}.
