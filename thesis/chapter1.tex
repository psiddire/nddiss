%
% Chapter 1
%

\chapter{Introduction}

The nature of our universe is explored in particle physics by understanding the properties of elementary particles and the fundamental forces between them. The Standard Model (SM) of particle physics has been well-tested, and it is consistent with all known particles and all known interactions, but gravity. The fundamental forces are mediated by the exchange of particles, called gauge bosons. The \PW and \PZ bosons mediate the weak interaction. These bosons are heavy, and they acquire mass through the Higgs mechanism. The Higgs mechanism is incorporated in SM by introducing a new scalar field associated with a new particle, the Higgs boson.

Yukawa interactions between the scalar field and the fermion fields give rise to the mass of the fermions. In SM, the mass matrix and the Yukawa interaction matrix are diagonalized simultaneously by choosing a particular eigenstate basis. This corresponds to the SM Higgs boson not having lepton flavor violating (LFV) decays. The SM has been well-tested over a wide range of energies, and despite its success, it is considered an effective theory as it fails to include gravity. It also cannot explain Dark Matter, which is inferred from astrophysical and cosmological observations. It also cannot explain Dark Energy, which is inferred to explain our universe's accelerated expansion.

Flavor physics is an active area of research as we try to understand the reason for observing six quarks and six fermions that can be arranged into three generations. The large difference between matter and anti-matter in our universe cannot be explained completely by the SM's CP-violating effects. Cabibbo-Kobayashi-Maskawa (CKM) matrix is the observed CP-violation source, a $3 \times 3$ unitary matrix that can be parameterized by three mixing angles and one complex CP-violating phase \cite{Tanabashi:2018oca}. The CKM matrix describes the quarks' mixing because the mass eigenstates are not equal to the weak eigenstates.

Due to the absence of right-handed neutrinos, there is no Dirac mass term for neutrinos in the SM. There is no Majorana mass term due to the exact conservation of baryon and lepton number. However, several experiments have observed neutrino oscillations, which can be explained by nonzero neutrino masses. The mixing of the neutrinos is described by the Pontecorvo-Maki-Nakagawa-Sakata (PMNS) matrix \cite{Tanabashi:2018oca}. An additional source of the CP-violation is the PMNS matrix described by three mixing angles and a phase. Besides, the SM cannot explain the substantial difference between the very small neutrino masses ($m < 2 \eV$) and the masses of the charged leptons and quarks ($\sim 173 \GeV$ for the top quark) \cite{Tanabashi:2018oca}.

One question that needs to be addressed is any relation between quark mixing and neutrino mixing. The Yukawa interactions within the SM cannot be constrained by Gauge invariance. To address the open questions, we need new physics models. Grand Unification Theories (GUTs) are a class of new physics models that try to unify all known interactions but gravity in a single gauge group. At the same time, extra dimensions could link flavor to the geometry of these extra dimensions \cite{Raidal:2008jk}. Another class of new physics models is simple extensions of the SM with more than one Higgs doublets, and they could add additional sources for CP-violation. These multi-Higgs doublet models can have tree level Higgs-mediated flavor changing neutral currents leading to lepton-flavor violating (LFV) Yukawa couplings \cite{Raidal:2008jk}.

Large Hadron Collider (LHC) \cite{Evans:2008zzb} was built to discover the Higgs boson and other exotic particles predicted by models like Supersymmetry or potential Dark Matter particles like weakly interacting massive particles. It was designed to run at a center-of-mass energy of $\sqs = 14 \TeV$ and is the most powerful particle collider built and operational to date. ATLAS and Compact Muon Solenoid (CMS) are two general-purpose detectors designed to detect the result of the particle collisions. In 2012, both experiments discovered the Higgs boson with a mass of $125 \GeV$ \cite{Aad:2012tfa, Chatrchyan:2012ufa}. Many precision measurements need to be performed to confirm if it is the SM Higgs boson or a Higgs boson of an SM's possible extension. Until now, no significant deviations from the SM Higgs boson have been observed.

In this thesis, a search for the Higgs boson's LFV decays to a muon and a tau (\mutau) or an electron and a tau (\etau) is presented. In each channel, the tau can further decay either hadronically or leptonically. If the tau lepton decays leptonically, we only consider final states with different lepton flavors to avoid the large Drell-Yan background. Thus, the \mutau channel is further divided into \muhad and \mue final states, while the \etau channel is further divided into the \ehad and \emu final states. This search is performed with proton-proton collision data collected at a center-of-mass energy of $\sqs = 13 \TeV$ in 2016, 2017, and 2018 corresponding to an integrated luminosity of $35.9 \fb$, $41.5 \fb$, and $59.3 \fb$, respectively. Thus, the total integrated luminosity analyzed in this search is $137 \fb$, which corresponds to a four times larger dataset than the one used in the previous search.

Apart from the much larger dataset, significant improvements have been made regarding the background estimation techniques and a detailed study of the systematic uncertainties involved in the analysis. A significant portion of the background is estimated using data-driven techniques with limited dependence on the MC simulations. This gives rise to a better description of the event kinematics and an improvement with regards to the corresponding systematics involved. The categorization of the events has been kept the same as the previous search. In contrast, improvements have been made to the classification done with multivariate techniques for discriminating the signal from the background to improve the search's sensitivity. All these changes collectively gave rise to a factor of two improvements in the sensitivity of the search and helped to set the most stringent limits set on these LFV Higgs decays to date.

This thesis is structured as follows. An overview of the SM of particle physics, along with a short review on LFV decays of the Higgs boson, is given in Chapter 2. In that chapter, we will see how new physics can introduce LFV Yukawa-couplings and how low-energy measurements constrain them \cite{Harnik:2012pb}. The experimental setup of LHC and the CMS experiment is discussed in Chapter 3. In that chapter, the various sub-detectors of the CMS experiment, which work in unison, is explained. Chapter 4 will dive into the MC event generation, followed by the event reconstruction description in Chapter 5. The event selection is described in Chapter 6, followed by a detailed explanation of the background estimation in Chapter 7. We will understand the systematic uncertainties in Chapter 8 and the corresponding statistical analysis to obtain the results. We will then discuss the results in Chapter 9, followed by a conclusion. Some studies for the future LFV analysis have been performed, and they are detailed in the Appendix.
